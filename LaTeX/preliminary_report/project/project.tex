\chapter{Inverse of Upper Banded Matrix}



\section{Task}

\noindent Given an upper banded matrix $U$, find $X$ and $Y$ such that $U^{-1} = \triu(X Y^{\T})$



\section{Algorithm}

\noindent For $U \in \mathbb{R}^{n \times n}$ with banded width $u$, \\
if $n = ku$, then
\begin{align*}
    X &\coloneqq U^{-1} E_{k}; \\
    Y_{i}^{\T} &\coloneqq (D_{i} X_{i})^{-1}\ \text{for} \ i = 1,\, 2,\, \dots,\, k;
\end{align*}

\noindent otherwise $n = ku+r$, and
\begin{align*}
    X &\coloneqq U^{-1} E_{k}; \\
    Y_{i}^{\T} &\coloneqq (D_{i} X_{i})^{-1}\ \text{for} \ i = 1,\, 2,\, \dots,\, k;\\
    \tilde{Y_{0}}^{\T} &\coloneqq (\tilde{D_{0}} \tilde{X_{0}})^{+} = (\tilde{D_{0}} \tilde{X_{0}})^{\T} (\tilde{D_{0}} \tilde{X_{0}} (\tilde{D_{0}} \tilde{X_{0}})^{\T})^{-1}.
\end{align*}

\newpage
\subsection{Pseudocode}

\IncMargin{1em}
\begin{algorithm}
    \SetKwFunction{backsub}{BackSubstitution}
    \SetKwInOut{Input}{Input}\SetKwInOut{Output}{Output}
    \Input{An upper banded matrix $U \in \mathbb{R}^{n \times n}$ and its bandwidth $u$.}
    \Output{Its inverse $U^{-1}$ and $X,Y \in \mathbb{R}^{n \times u}$ such that $U^{-1} = \triu(X Y^{\T})$.}
    \BlankLine
    \tcp{n = ku + r where 0 <= r < u}
    $k \leftarrow$ quotient of $n$ divided by $u$ 
    \tcp*{floor(n/u)}
    $r \leftarrow$ remainder of $n$ divided by $u$ 
    \tcp*{rem(n,u)}
    $E_{k} \leftarrow$ last block column of $I_{n \times n}$ 
    \tcp*{one(U)[:, n-u+1:n]}
    $X \leftarrow$ \backsub{$U$, $E_{k}$}\\
    \For{$i\leftarrow k$ \KwTo $1$}{
        $D_{i}\leftarrow i$-th diagonal block of $U$
        \tcp*{U[(i-1)*u+r+1:i*u+r, (i-1)*u+r+1:i*u+r]}
        $X_{i}\leftarrow i$-th block of $X$
        \tcp*{X[(i-1)*u+r+1:i*u+r, :]}
        $Y_{i}^{\T}\leftarrow (D_{i} X_{i})^{-1},\, i$-th block of $Y^{\T}$
        \tcp*{Yt[:,(i-1)*u+r+1:i*u+r]}
    }
    \eIf(\Return{$U^{-1} \coloneqq \triu(X Y^{\T})$}, $X$, $Y$){$r=0$}{}(need find $\tilde{Y}_{0}^{\T}$ using Moore-Penrose right inverse){
        $\tilde{D}_{0} \in \mathbb{R}^{r \times r} \leftarrow$ first diagonal block of $U$
        \tcp*{U[1:r, 1:r]}
        $\tilde{X}_{0}\in \mathbb{R}^{r \times u} \leftarrow$ first block of $X$
        \tcp*{X[1:r, :]}
        $\tilde{Y}_{0}^{\T}\in \mathbb{R}^{u \times r} \leftarrow (\tilde{D}_{0} \tilde{X}_{0})^{+} = (\tilde{D}_{0} \tilde{X}_{0})^{\T} (\tilde{D}_{0} \tilde{X}_{0} (\tilde{D}_{0} \tilde{X}_{0})^{\T})^{-1},$ first block of $Y^{\T}$
    }
    \Return{$U^{-1} \coloneqq \triu(X Y^{\T})$, $X$, $Y$}
    \caption{Inverse of banded matrix in outer product form}\label{Uinv_out_XY}
\end{algorithm}\DecMargin{1em}


\subsection{Example}

Let $U \in \mathbb{R}^{n \times n}$ and $n = 2u+r$, \textit{i.e.,} $k=2$,

\begin{align*}
    U = 
    \begin{bNiceMatrix}[margin]
        \tilde{D_{0}}   & \tilde{B_{0}} &       \\
                & D_{1} & B_{1} \\
                &       & D_{2} 
    \end{bNiceMatrix}\in \mathbb{R}^{(2u+r) \times (2u+r)}
\end{align*}
\noindent where $\tilde{D_{0}} \in \mathbb{R}^{r \times r}$, $\tilde{B_{0}} \in \mathbb{R}^{r \times u}$, 
and $D_{1},\,B_{1},\,D_{2},\,B_{2},\, \in \mathbb{R}^{u \times u}$.


\noindent By definitions, $X \coloneqq U^{-1} E_{2}$ with $I=
\begin{bNiceMatrix}[vlines,margin]
    \tilde{E_{0}} & E_{1} & E_{2}
\end{bNiceMatrix}$, that is:



\NiceMatrixOptions{
    code-for-first-row = \scriptstyle,
    code-for-first-col = \scriptstyle,
    code-for-last-row = \scriptstyle,
    code-for-last-col = \scriptstyle,
    }
\begin{align*}
    UX &= E_{2} \\
    \begin{bNiceMatrix}[hlines,margin,first-row,first-col]
          & r & u & u \\
        r & \tilde{D_{0}}   & \tilde{B_{0}} & 0     \\
        u & 0     & D_{1} & B_{1} \\
        u & 0     & 0     & D_{2} 
    \end{bNiceMatrix}
    \begin{bNiceMatrix}[margin,first-row,first-col]
          & u  \\
        r & \tilde{X}_{0} \\
        u & X_{1} \\
        u & X_{2} 
    \end{bNiceMatrix}
    &=
    \begin{bNiceMatrix}[margin,first-row,first-col]
        & u  \\
        r & 0 \\
        u & 0 \\
        u & I 
    \end{bNiceMatrix}.
\end{align*}


\noindent Using back-subsitution to find $X$ as following:


\begin{align*}
    \begin{bNiceMatrix}[vlines,margin]
        0     & 0     & D_{2} 
    \end{bNiceMatrix}
    \begin{bNiceMatrix}[hlines,margin]
        \tilde{X}_{0} \\
        X_{1} \\
        X_{2} 
    \end{bNiceMatrix}
    &= D_{2} X_{2} = I_{u \times u} && X_{2} = D_{2}^{-1}\\
    \begin{bNiceMatrix}[vlines,margin]
        0     & D_{1} & B_{1}
    \end{bNiceMatrix}
    \begin{bNiceMatrix}[hlines,margin]
        \tilde{X}_{0} \\
        X_{1} \\
        X_{2} 
    \end{bNiceMatrix}
    &= D_{1} X_{1} + B_{1} X_{2} = 0_{u \times u} && X_{1} = -D_{1}^{-1} B_{1} X_{2} = -D_{1}^{-1} B_{1} D_{2}^{-1}\\
    \begin{bNiceMatrix}[vlines,margin]
        \tilde{D_{0}} & \tilde{B_{0}} & 0
    \end{bNiceMatrix}
    \begin{bNiceMatrix}[hlines,margin]
        \tilde{X}_{0} \\
        X_{1} \\
        X_{2} 
    \end{bNiceMatrix}
    &= \tilde{D_{0}} \tilde{X}_{0} + \tilde{B_{0}} X_{1} = 0_{r \times u} && \tilde{X}_{0} = -\tilde{D_{0}}^{-1} \tilde{B_{0}} X_{1} = \tilde{D_{0}}^{-1} \tilde{B_{0}} D_{1}^{-1} B_{1} D_{2}^{-1}\\
\end{align*}


\noindent Following the definition of $Y_{i}^{\T}$,

\begin{align*}
    Y_{2}^{\T} &= (D_{2} X_{2})^{-1} = X_{2}^{-1} D_{2}^{-1} = D_{2} D_{2}^{-1} = I_{u \times u} \\
    Y_{1}^{\T} &= (D_{1} X_{1})^{-1} = X_{1}^{-1} D_{1}^{-1} = -D_{2} B_{1}^{-1} D_{1} D_{1}^{-1} = -D_{2} B_{1}^{-1} \\
    \tilde{Y}_{0}^{\T} &= (\tilde{D}_{0} \tilde{X}_{0})^{+} = \tilde{X}_{0}^{+} \tilde{D}_{0}^{+}
\end{align*}

\noindent Note that $X_{2}$ and $Y_{2}^{\T}$ are upper triangular matrices, then $\triu(X_{2} Y_{2}^{\T}) = X_{2} Y_{2}^{\T}$.


\begin{align*}
    U \triu(X Y^{\T}) &= 
        \begin{bNiceMatrix}[hlines,margin]
            \tilde{D}_{0} & \tilde{B}_{0} & 0     \\
            0     & D_{1} & B_{1} \\
            0     & 0     & D_{2} 
        \end{bNiceMatrix}
        \begin{bNiceMatrix}[vlines,margin]
            \triu(\tilde{X}_{0} \tilde{Y}_{0}^{\T}) & \tilde{X}_{0} Y_{1}^{\T}        & \tilde{X}_{0} Y_{2}^{\T} \\
            0                       & \triu(X_{1} Y_{1}^{\T}) & X_{1} Y_{2}^{\T} \\
            0                       & 0                         & X_{2} Y_{2}^{\T} 
        \end{bNiceMatrix}\\
        &= 
        \begin{bNiceMatrix}[hlines,margin]
            \tilde{D}_{0} & \tilde{B}_{0} & 0     \\
            0     & D_{1} & B_{1} \\
            0     & 0     & D_{2} 
        \end{bNiceMatrix}
        \begin{bNiceMatrix}[vlines,margin]
            \triu(\tilde{D}_{0}^{+}) & -\tilde{D}_{0}^{-1} \tilde{D}_{0} D_{1}^{-1} & \tilde{D_{0}}^{-1} \tilde{B_{0}} D_{1}^{-1} B_{1} D_{2}^{-1} \\
            0                & \triu(D_{1}^{-1})            & -D_{1}^{-1} B_{1} D_{2}^{-1} \\
            0                & 0                            & D_{2}^{-1}
        \end{bNiceMatrix}\\
        &=
        \begin{bNiceMatrix}[hlines,margin]
            \tilde{D}_{0} & \tilde{B}_{0} & 0     \\
            0     & D_{1} & B_{1} \\
            0     & 0     & D_{2} 
        \end{bNiceMatrix}
        \begin{bNiceMatrix}[vlines,margin]
            \tilde{D}_{0}^{-1} & -\tilde{D}_{0}^{-1} \tilde{D}_{0} D_{1}^{-1} & \tilde{D_{0}}^{-1} \tilde{B_{0}} D_{1}^{-1} B_{1} D_{2}^{-1} \\
            0                & D_{1}^{-1}           & -D_{1}^{-1} B_{1} D_{2}^{-1} \\
            0                & 0                            & D_{2}^{-1}
        \end{bNiceMatrix}\\
        &=
        \begin{bNiceMatrix}[vlines,margin]
            \tilde{D}_{0} \tilde{D}_{0}^{-1} & -\tilde{B}_{0} D_{1}^{-1}+\tilde{B}_{0} D_{1}^{-1} & \tilde{B_{0}} D_{1}^{-1} B_{1} D_{2}^{-1}-\tilde{B_{0}} D_{1}^{-1} B_{1} D_{2}^{-1} \\
            0          & D_{1} D_{1}^{-1}                   & -B_{1} D_{2}^{-1}+B_{1} D_{2}^{-1} \\
            0          & 0                            & D_{2} D_{2}^{-1}
        \end{bNiceMatrix}\\
        &=
        \begin{bNiceMatrix}[vlines,margin]
            I_{r \times r} & 0 & 0 \\
            0 & I_{u \times u} & 0 \\
            0 & 0 & I_{u \times u} 
        \end{bNiceMatrix}\\
        &= I_{n \times n}
\end{align*}




\section{Derivation}
Consider an upper banded matrix $U \in \mathbb{R}^{n \times n}$ with banded width $u$,

\begin{align}
    U = 
    \begin{bNiceMatrix}[margin]
        \alpha_{1} & \beta^{1}_{1} & \beta^{2}_{1} & \Cdots & \beta^{u}_{1} &   &   &   \\
          & \alpha_{2} & \beta^{1}_{2} & \beta^{2}_{2} & \Cdots & \beta^{u}_{2} &   &   \\
          &   & \Ddots & \Ddots &  &   & \Ddots &   \\
          &   &   &   &   &   &   & \beta^{u}_{n-u} \\
          &   &   &   &   &   &   & \Vdots \\
          &   &   &   &   &   &   & \beta^{2}_{n-2} \\
          &   &   &   &   &   &   & \beta^{1}_{n-1} \\
          &   &   &   &   &   &   & \alpha_{n} \\
    \end{bNiceMatrix}\label{U_banded}.
\end{align}
% \nonumber

\noindent Then try to show that its inverse can be repersent as the upper triangular part of an outer product,
\textit{i.e.}\ ,
\begin{align}
    U^{-1} = \triu(XY^{\T}) \label{main_eqn}
\end{align}

\noindent Let $X,\ Y \in \mathbb{R}^{n \times u}$, consider the following cases of $n$,

\subsection*{Case 1, \bm{$n = ku$}}

then \eqref{U_banded} can be repersent as a block upper bidiagonal matrix,


\NiceMatrixOptions{code-for-first-row = \scriptstyle,code-for-first-col = \scriptstyle}
\setcounter{MaxMatrixCols}{30}
\begin{align}
    U &=
    \begin{bNiceMatrix}[first-row,first-col,margin]
          & \hspace*{1mm} & \Cdots[line-style={solid,<->}]^{\begin{turn}{0}$u$\end{turn}} & \hspace*{1mm} &
            \hspace*{1mm} & \Cdots[line-style={solid,<->}]^{\begin{turn}{0}$u$\end{turn}} & \hspace*{1mm} &
            \hspace*{1mm} & \Cdots & \hspace*{1mm} &
            \hspace*{1mm} & \Cdots[line-style={solid,<->}]^{\begin{turn}{0}$u$\end{turn}} & \hspace*{1mm} \\
        % 
        \hspace*{1mm} &
            \alpha_{1} & \Cdots & \beta^{u-1}_{1} & \beta^{u}_{1} &&&&&&&&\\
        \Vdots[line-style={solid,<->}]_{\begin{sideways}$u$\end{sideways}} &
            & \Ddots & \Vdots & \Vdots & \Ddots &&&&&&&\\
        \hspace*{1mm} &
            && \alpha_{u} & \beta^{1}_{u} & \Cdots & \beta^{u}_{u} &&&&&&\\
        % 
        \hspace*{1mm} &
            &&& \alpha_{u+1} & \Cdots & \beta^{u-1}_{u+1} & \beta^{u}_{u+1} &&&&&\\
        \Vdots[line-style={solid,<->}]_{\begin{sideways}$u$\end{sideways}} &
            &&&& \Ddots & \Vdots & \Vdots & \Ddots &&&&\\
        \hspace*{1mm} &
            &&&&& \alpha_{2u} & \beta^{1}_{2u} & \Cdots & \beta^{u}_{2u} &&&\\
        % 
        \hspace*{1mm} &
            &&&&&&&&&&&\\
        \Vdots &
            &&&&&&  & \ddots &\ddots&\ddots& \ddots &\\
        \hspace*{1mm} &
            &&&&&&&&&&&\\
        % 
        \hspace*{1mm} &
            &&&&&&&&& \alpha_{(k-1)u+1} & \Cdots & \beta^{u-1}_{(k-1)u+1} \\
        \Vdots[line-style={solid,<->}]_{\begin{sideways}$u$\end{sideways}} &
            &&&&&&&&&& \Ddots & \Vdots \\
        \hspace*{1mm} &
            &&&&&&&&&&& \alpha_{ku}\\
    \end{bNiceMatrix}
    \nonumber \\
    &= 
    \begin{bNiceMatrix}[margin]
        D_{1}   & B_{1} &           &           &  \\
                & D_{2} & B_{2}     &           &  \\
                &       & \ddots    & \ddots    &  \\
                &       &           & D_{k-1}   & B_{k-1}\\
                &       &           &           & D_{k}
    \end{bNiceMatrix}\in \mathbb{R}^{ku \times ku},\, D_{i},\,B_{i} \in \mathbb{R}^{u \times u} \ \text{for} \ i =  1,\, 2,\, \dots,\, k.
    \label{U_block_bi}
\end{align}

\noindent Repersent the $n$-dimensional identity matrix $I_{n \times n}$ in block form,

\begin{align*}
    I_{n \times n} = I_{ku \times ku} = \diag(\underbrace{I_{u \times u},\ I_{u \times u},\ \cdots,\ I_{u \times u}}_{k}) =
    \begin{bNiceMatrix}[vlines,margin]
        E_{1} & E_{2} & \cdots & E_{k}
    \end{bNiceMatrix}.
\end{align*}


\noindent then, its block columns $E_{i}$ for $i =  1,\, 2,\, \dots,\, k$ satisfies

\begin{align*}
    E_{i} &\in \mathbb{R}^{ku \times u} \\
    E_{i}^{\T} E_{j} &= 
    \begin{cases}
        I_{u \times u}, & \text{if} \ i=j \ ;\\
        0_{u \times u}, & \text{otherwise}.
    \end{cases}.
\end{align*}


\noindent To find $X$ and $Y$ such that \eqref{main_eqn} holds, let

\begin{align*}
    X =
    \begin{bNiceMatrix}[margin]
        X_{1} \\
        X_{2} \\
        \vdots \\
        X_{k-1} \\
        X_{k}
    \end{bNiceMatrix} \in \mathbb{R}^{ku \times u},\,
    Y = 
    \begin{bNiceMatrix}[margin]
        Y_{1} \\
        Y_{2} \\
        \vdots \\
        Y_{k-1} \\
        Y_{k}
    \end{bNiceMatrix} \in \mathbb{R}^{ku \times u},\, X_{i},\,Y_{i} \in \mathbb{R}^{u \times u}\ \text{for} \ i =  1,\, 2,\, \dots,\, k,
\end{align*}

\noindent and let $X_{k}$ and $Y_{k}^{\T}$ be upper triangular matrix, 
then $\triu(X_{k} Y_{k}^{\T}) = X_{k} Y_{k}^{\T}$.


\noindent For the outer product form, 

\begin{align*}
    Y^{\T} &=
        \begin{bNiceMatrix}[vlines,margin]
            Y_{1}^{\T} & Y_{2}^{\T} & \cdots & Y_{k-1}^{\T} & Y_{k}^{\T}
        \end{bNiceMatrix} \in \mathbb{R}^{u \times ku} \\
    XY^{\T} &=
        \begin{bNiceMatrix}[margin]
            X_{1} \\
            X_{2} \\
            \vdots \\
            X_{k-1} \\
            X_{k}
        \end{bNiceMatrix}
        \begin{bNiceMatrix}[vlines,margin]
            Y_{1}^{\T} & Y_{2}^{\T} & \cdots & Y_{k-1}^{\T} & Y_{k}^{\T}
        \end{bNiceMatrix} \\
        &= \begin{bNiceMatrix}[vlines,margin]
            X_{1} Y_{1}^{\T} & X_{1} Y_{2}^{\T} & \cdots & X_{1} Y_{k-1}^{\T} & x_{1} Y_{k}^{\T}\\
            X_{2} Y_{1}^{\T} & X_{2} Y_{2}^{\T} & \cdots & X_{2} Y_{k-1}^{\T} & x_{2} Y_{k}^{\T}\\
            \vdots & \vdots & \vdots & \vdots & \vdots\\
            X_{k-1} Y_{1}^{\T} & X_{k-1} Y_{2}^{\T} & \cdots & X_{k-1} Y_{k-1}^{\T} & X_{k-1} Y_{k}^{\T}\\
            X_{k} Y_{1}^{\T} & X_{k} Y_{2}^{\T} & \cdots & X_{k} Y_{k-1}^{\T} & X_{k} Y_{k}^{\T}
        \end{bNiceMatrix} \\
        &= X
        \begin{bNiceMatrix}[vlines,margin]
            Y_{1}^{\T} & Y_{2}^{\T} & \cdots & Y_{k-1}^{\T} & Y_{k}^{\T}
        \end{bNiceMatrix} \\
        &= 
        \begin{bNiceMatrix}[vlines,margin]
            XY_{1}^{\T} & XY_{2}^{\T} & \cdots & XY_{k-1}^{\T} & XY_{k}^{\T}
        \end{bNiceMatrix} \in \mathbb{R}^{ku \times ku},
\end{align*}

\noindent and its the upper triangular componnent,

\begin{align}
    \triu(XY^{\T}) &= 
        \begin{bNiceMatrix}[vlines,margin]
            \triu(X_{1} Y_{1}^{\T}) & X_{1} Y_{2}^{\T} & \cdots & X_{1} Y_{k-1}^{\T} & X_{1} Y_{k}^{\T}\\
            0_{u \times u} & \triu(X_{2} Y_{2}^{\T}) & \cdots & X_{2} Y_{k-1}^{\T} & X_{2} Y_{k}^{\T}\\
            \vdots & \vdots & \vdots & \vdots & \vdots\\
            0_{u \times u} & 0_{u \times u} & \cdots & \triu(X_{k-1} Y_{k-1}^{\T}) & X_{k-1} Y_{k}^{\T}\\
            0_{u \times u} & 0_{u \times u} & \cdots & 0_{u \times u} & X_{k} Y_{k}^{\T}
        \end{bNiceMatrix}
        \nonumber \\
        &= 
        \begin{bNiceMatrix}[vlines,margin]
            \triu(X_{1} Y_{1}^{\T}) & X_{1} Y_{2}^{\T}          &        & X_{1:k-2}Y_{k-1}^{\T}      &  \\
            0_{(k-2)u \times u}     & \triu(X_{2} Y_{2}^{\T})   & \cdots & \triu(X_{k-1}Y_{k-1}^{\T}) & X Y_{k}^{\T}\\
            0_{u \times u}          & 0_{(k-2)u \times u}       &        & 0_{u \times u}             &  
        \end{bNiceMatrix}. \label{triuXYt}
\end{align}


\noindent From \eqref{main_eqn},

\begin{align}
    I_{n \times n} = U \triu(XY^{\T}). \label{focus_eqn}
\end{align}


\noindent For $X$ and $Y_{k}$, from \eqref{triuXYt}, the $k$-th block column of \eqref{focus_eqn} can be written as,

\begin{align}
    E_{k} &= U \triu(XY^{\T})E_{k} \nonumber \\
    E_{k} &= U X Y_{k}^{\T} \label{eqnX_ku}
\end{align}


\noindent Define that

\begin{align}
    X &\coloneqq U^{-1} E_{k}; \label{X_ku} \\ 
    Y_{k}^{\T} &\coloneqq I_{u \times u}, \label{Yk_ku}
\end{align}

\noindent which can be shown that \eqref{eqnX_ku} holds for the defined \eqref{X_ku} and \eqref{Yk_ku} through

\begin{align*}
    U X Y_{k}^{\T} &= U U^{-1} E_{k} I_{u \times u} \\
        &= I_{n \times n} E_{k} I_{u \times u} \\
        &= E_{k} I_{u \times u} \\
        &= E_{k},\ \text{as required}.
\end{align*}


\noindent Then, for $Y_{k-1}$, forming the $(k-1)$-th block column of \eqref{focus_eqn} from \eqref{triuXYt} first,

\begin{align*}
    E_{k-1} &= U \triu(XY^{\T})E_{k-1} \\
    &= U 
    \begin{bNiceMatrix}[margin]
        X_{1:k-2}Y_{k-1}^{\T} \\
        \triu(X_{k-1}Y_{k-1}^{\T}) \\
        0_{u \times u}
    \end{bNiceMatrix}.
\end{align*}

\noindent Focusing on the $(k-1)$-th block row of $(k-1)$-th block column, 


% \begin{bNiceMatrix}[margin]
%     X_{1:k-2}Y_{k-1}^{\T} \\
%     \triu(X_{k-1}Y_{k-1}^{\T}) \\
%     0_{u \times u}
% \end{bNiceMatrix}

\begin{align}
    E_{k-1}^{\T} E_{k-1} &= E_{k-1}^{\T} U 
        \begin{bNiceMatrix}[margin]
            X_{1:k-2}Y_{k-1}^{\T} \\
            \triu(X_{k-1}Y_{k-1}^{\T}) \\
            0_{u \times u}
        \end{bNiceMatrix} \nonumber \\
    I_{u \times u} &=
        \begin{bNiceMatrix}[vlines,margin]
            0_{u \times (k-2)u} & D_{k-1} & B_{k-1}
        \end{bNiceMatrix} 
        \begin{bNiceMatrix}[hlines,margin]
            X_{1:k-2}Y_{k-1}^{\T} \\
            \triu(X_{k-1}Y_{k-1}^{\T}) \\
            0_{u \times u}
        \end{bNiceMatrix} \nonumber \\
    I_{u \times u} &= D_{k-1} \triu(X_{k-1}Y_{k-1}^{\T}). \label{eqnY_k_1u}
\end{align}



\noindent Define that

\begin{align}
    Y_{k-1}^{\T} &\coloneqq (D_{k-1} X_{k-1})^{-1}, \label{eqn_Y_k_1_ku}
\end{align}

\noindent and notice that $D_{k-1} \in \mathbb{R}^{u \times u}$ is a upper triangular, therefore 

\begin{align}
    D_{k-1} &= \triu(D_{k-1}); \nonumber \\
    D_{k-1}^{-1} &= \triu(D_{k-1}^{-1}). \label{D_k_1}
\end{align}

\noindent With the fact \eqref{D_k_1}, \eqref{eqnY_k_1u} holds for the defined \eqref{eqn_Y_k_1_ku} as, 

\begin{align*}
    D_{k-1} \triu(X_{k-1}Y_{k-1}^{\T}) &= D_{k-1} \triu(X_{k-1}(D_{k-1} X_{k-1})^{-1})\\
        &= D_{k-1} \triu(X_{k-1} X_{k-1}^{-1} D_{k-1}^{-1})\\
        &= D_{k-1} \triu(D_{k-1}^{-1})\\
        &= D_{k-1} D_{k-1}^{-1}\\
        &= I_{u \times u},\ \text{as required}.
\end{align*}

\noindent Similar ideas can be use to redefine $Y_{k}^{\T}$ in \eqref{Yk_ku}, from \eqref{X_ku},




\begin{align*}
    E_{k} &= U X \\
    E_{k}^{\T} E_{k} &= E_{k}^{\T} U X \\
    I_{u \times u} &= E_{k}^{\T} U X \\
        &= 
        \begin{bNiceMatrix}[vlines,margin]
            0_{u \times (k-1)u} & D_{k}
        \end{bNiceMatrix}
        \begin{bNiceMatrix}[hlines,margin]
            X_{1:k-1} \\
            X_{k}
        \end{bNiceMatrix}\\
        &= D_{k} X_{k} \\
    I_{u \times u} &= (D_{k} X_{k})^{-1},
\end{align*}

\noindent following that

\begin{align*}
    Y_{k}^{\T} &\coloneqq I_{u \times u} \\
        &\coloneqq (D_{k} X_{k})^{-1}.
\end{align*}

\noindent In summary, for $U \in \mathbb{R}^{ku \times ku}$, \eqref{main_eqn} holds the following definitions of $X$ and $Y$:

\begin{align}
    X &\coloneqq U^{-1} E_{k} \nonumber \\
    Y_{i}^{\T} &\coloneqq (D_{i} X_{i})^{-1} \label{Yti_blockbi}
\end{align}




\subsection*{Case 2, \bm{$n = ku+r$}}

In this case, \eqref{U_banded} can still be repersent as a block upper bidiagonal matrix,

\begin{align*}
    U &=
    \begin{bNiceMatrix}[first-row,first-col]
          & \hspace*{1mm} & \Cdots[line-style={solid,<->}]^{\begin{turn}{0}$r$\end{turn}} & \hspace*{1mm}
          & \hspace*{1mm} && \Cdots[line-style={solid,<->}]^{\begin{turn}{0}$u$\end{turn}} && \hspace*{1mm}
          & \hspace*{1mm} && \Cdots && \hspace*{1mm}
          & \hspace*{1mm} && \Cdots[line-style={solid,<->}]^{\begin{turn}{0}$u$\end{turn}} && \hspace*{1mm}\\
        %  
        \hspace*{1mm} &
        d_{1} & \Cdots & b^{r-1}_{1} & b^{r}_{1} & \Cdots & b^{u}_{1} &&&&&&&&&&&&\\
        \Vdots[line-style={solid,<->}]_{\begin{sideways}$r$\end{sideways}} &
        & \Ddots & \Vdots & \Vdots &&& \Ddots &&&&&&&&&&&\\
        \hspace*{1mm} &
        && d_{r} & b^{1}_{r} & \Cdots &&& b^{u}_{r} &&&&&&&&&&\\
        % 
        \hspace*{1mm} &
        &&& d_{1+r} & \Cdots &   &   & b^{u-1}_{1+r} & b^{u}_{r+1} &&&&&&&&&\\
          &&&&&&&&&&&&&&&&&&\\
        \Vdots[line-style={solid,<->}]_{\begin{sideways}$u$\end{sideways}} &
        &&&&& \Ddots && \Vdots & \Vdots && \Ddots &&&&&&&\\
          &&&&&&&&&&&&&&&&&&\\
        \hspace*{1mm} &
        &&&&&&& d_{u+r} & b^{1}_{u+r} && \Cdots && b^{u}_{u+r} &&&&&\\
        % 
        \hspace*{1mm} &
        &&&&&&&&&&&&&&&&&&\\
          &&&&&&&&&&&&&&&&&&\\
        \Vdots &
        &&&&&&&&&&& \ddots &&& \ddots &&&\\
          &&&&&&&&&&&&&&&&&&\\
        \hspace*{1mm} &
        &&&&&&&&&&&&&&&&&&\\
        % 
        \hspace*{1mm} &
        &&&&&&&&&&&&& d_{n-u+1} && \Cdots && b^{u}_{n-u+1}\\
         &&&&&&&&&&&&&&&&&&  \\
        \Vdots[line-style={solid,<->}]_{\begin{sideways}$u$\end{sideways}} &
        &&&&&&&&&&&&&&& \Ddots && \Vdots \\
         &&&&&&&&&&&&&&&&&&  \\
        \hspace*{1mm} &
        &&&&&&&&&&&&&&&&& d_{n}\\
    \end{bNiceMatrix} \\
    &= 
    \begin{bNiceMatrix}[margin]
        \tilde{D}_{0}   & \tilde{B}_{0} &           &           &  \\
                & D_{1} & B_{1}     &           &  \\
                &       & \ddots    & \ddots    &  \\
                &       &           & D_{k-1}   & B_{k-1}\\
                &       &           &           & D_{k}
    \end{bNiceMatrix}\in \mathbb{R}^{(ku+r) \times (ku+r)},
\end{align*}


\noindent where $\, \tilde{D}_{0} \in \mathbb{R}^{r \times r},\,\tilde{B}_{0} \in \mathbb{R}^{r \times u},\, D_{i},\,B_{i} \in \mathbb{R}^{u \times u} \ \text{for} \ i =  1,\, 2,\, \dots,\, k.$\\
\noindent Then the partitions of $X$, $Y$, and $I_{n \times n}$ changed as $n$ cannot be divided by $u$,


\begin{align*}
    X =
    \begin{bNiceMatrix}[margin]
        \tilde{X}_{0} \\
        X_{1} \\
        \vdots \\
        X_{k}
    \end{bNiceMatrix} \in \mathbb{R}^{(ku+r) \times u},\,
    Y = 
    \begin{bNiceMatrix}[margin]
        \tilde{Y}_{0} \\
        Y_{1} \\
        \vdots \\
        Y_{k}
    \end{bNiceMatrix} \in \mathbb{R}^{(ku+r) \times u},\, \tilde{X}_{0},\,\tilde{Y}_{0} \in \mathbb{R}^{r \times u},\, X_{i},\,Y_{i} \in \mathbb{R}^{u \times u}\ \text{for} \ i =  1,\, 2,\, \dots,\, k,
\end{align*}


\noindent and


\begin{align*}
    I_{n \times n} = I_{(ku+r) \times (ku+r)} = \diag(I_{r \times r},\ \underbrace{I_{u \times u},\ \cdots,\ I_{u \times u}}_{k}) =
    \begin{bNiceMatrix}[vlines,margin]
        \tilde{E}_{0} & E_{1} & \cdots & E_{k}
    \end{bNiceMatrix}.
\end{align*}


\noindent Following the above statement, \eqref{Yti_blockbi} cannot be defined for case $i=0$ as $\tilde{D}_{0} \tilde{X}_{0}$ is a rectangular matrix which cannot be invertible.
However, this can be solved easily by the introduction of the Moore-Penrose as follows:


\begin{align*}
    \tilde{Y}_{0}^{\T} &\coloneqq (\tilde{D}_{0} \tilde{X}_{0})^{+} \\
        &\coloneqq (\tilde{D_{0}} \tilde{X_{0}})^{\T} (\tilde{D_{0}} \tilde{X_{0}} (\tilde{D_{0}} \tilde{X_{0}})^{\T})^{-1} \\
    \tilde{D}_{0} \triu(\tilde{X_{0}} \tilde{Y_{0}}^{\T}) &= \tilde{D}_{0} \triu(\tilde{X_{0}} (\tilde{D}_{0} \tilde{X}_{0})^{+}) \\
        &= \tilde{D}_{0} \triu(\tilde{X_{0}} \tilde{X}_{0}^{+} \tilde{D}_{0})^{+})\\
        &= \tilde{D}_{0} \triu(\tilde{D}_{0}^{+})\\
        &= \tilde{D}_{0} \triu(\tilde{D}_{0}^{-1})\\
        &= \tilde{D}_{0} \tilde{D}_{0}^{-1}\\
        &= I_{r \times r},\ \text{as required}.
\end{align*}


In conclusion, $X$ and $Y$ can be defined as following,

\begin{align*}
    X &\coloneqq U^{-1} E_{k}; \\
    Y_{i}^{\T} &\coloneqq 
    \begin{cases}
        (D_{i} X_{i})^{-1}\ \text{for} \ i = 1,\, 2,\, \dots,\, k, & \text{if} \ n=ku \ ;\\
        \begin{cases}
            (D_{i} X_{i})^{-1} & \text{if} \ i = 1,\, 2,\, \dots,\, k ;\\
            (\tilde{D}_{0} \tilde{X}_{0})^{+}  & \text{otherwise}.
        \end{cases}
        & \text{otherwise}.
    \end{cases}
    % (D_{i} X_{i})^{-1}\ \text{for} \ i = 1,\, 2,\, \dots,\, k.
\end{align*}
















